\documentclass{article}
\usepackage[utf8]{inputenc}
\usepackage{amsmath}
\usepackage{braket}
\usepackage{gensymb}
\usepackage{amssymb}
\usepackage{natbib}
\usepackage{graphicx}
\usepackage{listings}
\usepackage{color}
\usepackage{tikz}
\usepackage{multicol}
\usetikzlibrary{arrows}
\usepackage{hyperref}
\hypersetup{
    colorlinks=true,
    linkcolor=blue,
    filecolor=magenta,      
    urlcolor=cyan,
}
\usepackage{float}
\restylefloat{figure}

\usepackage[figurename=Figure]{caption}

\definecolor{codegreen}{rgb}{0,0.6,0}
\definecolor{codegray}{rgb}{0.5,0.5,0.5}
\definecolor{codepurple}{rgb}{0.58,0,0.82}
\definecolor{backcolour}{rgb}{0.95,0.95,0.92}
 
\lstdefinestyle{mystyle}{
    backgroundcolor=\color{backcolour},   
    commentstyle=\color{codegreen},
    keywordstyle=\color{magenta},
    numberstyle=\tiny\color{codegray},
    stringstyle=\color{codepurple},
    basicstyle=\footnotesize,
    breakatwhitespace=false,         
    breaklines=true,                 
    captionpos=b,                    
    keepspaces=true,                 
    numbers=left,                    
    numbersep=5pt,                  
    showspaces=false,                
    showstringspaces=false,
    showtabs=false,                  
    tabsize=2
}
 
\lstset{style=mystyle}
\lstset{
    language=Erlang,
    mathescape=true
}

\title{FYS3140 - Home exam 2018}
\author{Candidate: 15028}
\date{April 2018}

\begin{document}

\maketitle

\section*{Problem 1: Differential equation}


We are asked to solve the differential equation
\begin{equation}
y^{''}(x) + \frac{3}{x}y^{'}(x) - \frac{24}{x^2}y(x) = 56x^6.
\end{equation}
Multiplying through by $x^2$ yields
\begin{equation}
x^2y^{''}(x) + 3xy^{'}(x) - 24y(x) = 56x^8,
\end{equation}
which has the form $ax^2y^{''} + bxy^{y} + cy = g(x)$ and thus is a second order non-homogeneous Cauchy-Euler differential equation.

Solving this equation involves two major steps; 1) find the complementary function $f_c$, and 2) find the particular solution.

For the complementary function, we reconize that $a = 1, b = 3$ and $c = -24$, and write 
\begin{equation}
am(m-1) + bm + c = 0 \rightarrow m(m-1) + 3m -24 = m^2 + 2m -24 = 0,
\end{equation}
which yields $m_1 = 4$ and $m_2 = -6$. Since $m_1$ and $m_2$ are two distinct real roots the complementary function is a function on the form
\begin{equation}
y_c = c_1x^{m_1} + c_2x^{m^2} \rightarrow y_c = c_1x^{4} + c_2x^{-6}.
\end{equation}

For the particular solution we will use the variation of parameters. Given $y^{''} + p(x)y^{'} + q(x)y = g(x)$. Since $p(x) = 3/x, q(x) = -24/x^2$ and $g(x) = 56x^2$ are all continous on an open interval, the particular solution can be found by
\begin{equation}
Y_p = -y_1\int \frac{y_2g(x)}{W(y_1,y_2)}dx + y_2\int \frac{y_1g(x)}{W(y_1, y_2)}dx,
\end{equation}
where $W(y_1, y_2)$ is the Wronskian of $y_1$ and $y_2$. $y_1$ and $y_2$ is from the complementary function. Starting by finding the Wronskian of $y_1$ and $y_2$
\begin{equation}
W = \begin{vmatrix}y_1&y_2\\y_1^{'}&y_2^{'}\end{vmatrix} \rightarrow W = \begin{vmatrix}x^4&x^{-6}\\4x^3&-6x^{-7}\end{vmatrix} = -6x^{-7}x^4 - 4x^3x^{-6} = -10x^{-3},
\end{equation}
we can write
\begin{align}
Y_p &= -x^4 \int \frac{x^{-6}56x^6}{-10x^{-3}}dx + x^{-6} \int \frac{x^456x^6}{-10x^{-3}}dx \\
 &= \frac{56}{10}\bigg(x^4 \int x^3 \ dx - x^{-6} \int x^{13} \ dx \bigg) \\
 &= \frac{56}{10}\bigg(\frac{x^8}{4} - \frac{x^{8}}{14}\bigg) \\
 &= \frac{56}{10}\bigg(\frac{10x^8}{56}\bigg) \\
 &= x^8
\end{align}

Finnaly, we find our general solution by adding the complementary function and the particular solution togetter
\begin{equation}
y(x) = y_c + Y_P \rightarrow y(x) = c_1x^4 + c_2x^{-6} + x^8,
\end{equation}
which also can be written as
\begin{equation}
y(x) = \frac{c_2}{x^{6}} + c_1x^4 + x^8,
\end{equation}
and that's my final answer.

\section*{Problem 2: Complex analysis}

\subsection*{Part A:}

\subsubsection*{a$)$}

For a function that has a \textit{pole of order 3} at $z = 3 + i$, a \textit{zero of order 4} at $z = 2i$, we have the following function 
\begin{equation}
f(z) = \frac{(z-2i)^4}{(z-[3+i])^3}
\end{equation}

\subsubsection*{b$)$}

We are asked to classify the isolated singularity of the function
\begin{equation}
f(x) = \frac{z^3 + 8}{(z-5)^3(z+2)}.
\end{equation}
If we write
\begin{equation}
f(x) = \frac{1}{(z-5)^3}\frac{z^3+8}{(z+2)}.
\end{equation}
Polynomial division, $(z^3+8):(z+2)$, yields
\begin{equation}
f(x) = \frac{z^2 - 2z + 4}{(z-5)^3},
\end{equation}
which shows $z = -2$ is a \textit{removable singularity}. Now, if we write
\begin{align}
\frac{1}{(z-5)^3} &= \bigg(\frac{1}{z-5}\bigg)^3 = \bigg(-\frac{\frac{1}{5}}{1-\frac{z}{5}}\bigg)^3 \\
 &= \bigg(-\frac{1}{5}\frac{1}{1-\frac{z}{5}}\bigg)^3 = \bigg(-\frac{1}{5}\sum_{n=0}^{\infty}\big(\frac{z}{5}\big)^n\bigg)^3 \\
 &= \bigg(-\frac{1}{5}\bigg[1 + \frac{z}{5} + \frac{z^2}{25} + \frac{z^3}{25} + ... \bigg]\bigg)^3 \\
 &= \bigg(\bigg[-\frac{1}{5} - \frac{z}{25} - \frac{z^2}{125} - \frac{z^3}{675} + ... \bigg]\bigg)^3
\end{align}


\subsection*{Part B:}

\subsubsection*{a$)$}

\subsubsection*{b$)$}

\subsubsection*{c$)$}

\subsubsection*{d$)$}

\section*{Problem 3: The Dirac delta function}

\subsubsection*{a$)$}

\subsubsection*{b$)$}

\subsubsection*{c$)$}

\subsubsection*{a$)$}


\end{document}

